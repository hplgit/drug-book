% Commands that are used by several input files can be placed here

% Draws a figure similar to Fig 1A in Winslow2006.pdf
\newcommand{\winslow}[4]{
\begin{tikzpicture}[line width=0.3mm]
\tikzstyle{every node}=[font=\large]
% The cytol:
\draw[line width=0,fill=white!80!blue] (0,0) -- (12,0) -- (12,1) -- (2,1)  to [out=180, in=180] (2,2) -- (12,2) -- (12,7) -- (0,7) -- (0,0);
% The SR:
\draw[line width=0,fill=white!80!green] (0,9) -- (12,9) -- (12,7) -- (6,7) -- (6,6) -- (7,6) to [out=0, in=0] (7,4) -- (4,4) to [out=180, in=180] (4,6) --   (5,6)   -- (5,7) -- (0,7) -- (0,9);
\node  at    (10,8) {NSR};
\node  at    (4,5) {JSR};
% The Dyad box:
\draw[black, dashed, rounded corners, fill=white!80!red] (3.8,2.1) rectangle (7.2,3.9);
\node at ( 5.6,3.3) {Dyad};
\node at (1.1,3.3) {Cytosol};
% Current from NSR to JSR:
\draw[->, red] (5.5,6.6) -- (5.5,5);
\node[red] at (6.1,5.5) { $\mbox{Ca}^{2+}$};
% Current from JSR to the Dyad:
\foreach \i in {0,...,12}
{
\draw[->, black!30!blue] (4+0.25*\i,4.2) -- (4+0.25*\i,3.7);
}
\node[black!30!blue]  at (7.8,4) {RyRs};
% Current into Dyad:
\draw[->, red] (4.5,2.5) -- (4.5,3);
\draw[->, red] (6.5,2.5) -- (6.5,3);
\node[red] at (5.1,2.8) { $\mbox{Ca}^{2+}$};
% Current out of Dyad:
\draw[->, red] (4.2,3.1) to [out=270, in=0]  (2,2.5);
\draw[->, red] (6.8,3.1) to [out=270, in=180]  (9,2.5);
% The LCCs:
\draw[->, black!50!green] (4.5,1.6) -- (4.5,2.3);
\draw[->, black!50!green] (6.5,1.6) -- (6.5,2.3);
\node[black!50!green] at (5.1,1.8) {LCCs};
% Current in the T-tubule:
\draw[->, red] (11,1.4) -- (8.5,1.4);
\node[red] at (10,1.7) { $\mbox{Ca}^{2+}$};
\node  at (5,1.2) {T-tubule,\, extracellular space};
% Box that defines the region of interest, 1D, 2D or 3D
\draw[red] (#1,#2) rectangle (#3,#4);
\end{tikzpicture}
}
